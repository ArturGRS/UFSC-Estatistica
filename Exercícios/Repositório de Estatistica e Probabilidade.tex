\documentclass{article}
\usepackage[utf8]{inputenc}
\usepackage{amsmath, amssymb, amsthm}
\usepackage{graphicx, float}
\usepackage{tzplot}
\graphicspath{ {./imagens/} }
\usepackage{multicol}
\usepackage[brazil]{babel}
\usepackage{pgfplots}
\pgfplotsset{compat=1.18}
\usepackage[letterpaper, top = 1in, bottom = 1.0 in, left = 1.2 in, right = 1.2 in, heightrounded]{geometry}

%%%%%%%%%%%%%%%%%%%%%%%%% Caso haja dúvidas na Symbologia %%%%%%%%%%%%%%%%
% https://detexify.kirelabs.org/classify.html
%%%%%%%%%%%%%%%%%%%%%%%%% Parâmetros de construção %%%%%%%%%%%%%%%%%%%%%%%
\setlength{\parindent}{0pt}
\setlength{\parskip}{0.8em}

\title{\textbf{Repositório de Estatistica e Probabilidade}}
\author{UFSC Joinville - EMB5010 \\ Artur Gemaque}
\date{\today}
%%%%%%%%%%%%%%%%%%%%%%%%%% COMEÇO DO DOCUMENTO %%%%%%%%%%%%%%%%%%%%%%%%%%%
\begin{document}
\maketitle

\begin{abstract}
Este documento tempo como principal funcionalidade registrar os contudos ensinados 
em sala de aula pelo professor Jaimes, ademais servirá como fonte de estudo para as 
provas referentes há Matéria.
\end{abstract}

  \section{Teorema da Probabilidade Total e Teorema de Bayes}

    \begin{quote}
      Exemplo 1
    \end{quote}

    Um indivíduo possui 3 contas de e-mail diferentes. 
    \begin{multicols}{2}
      
    Do total de mensagens que ele recebe:
    
    \begin{itemize}
      \item 70\% na conta 1
      \item 20\% na conta 2
      \item 10\% na conta 3
    \end{itemize}

    Mensagens que são SPAM
    
    \begin{itemize}
      \item 1\% das  mensagens da conta 1 
      \item 2\% das  mensagens da conta 2
      \item 5\% das  mensagens da conta 3
    \end{itemize}

    \end{multicols}

    Questão:

    \begin{quote}
      1. Qual a Probabilidade de uma mensagem selecionada aleatoriamente ser SPAM ?
    \end{quote}
      
    Partindo que $ C_1 $ representa a probabilidade de uma mensagem chegar na conta 1 e $ S_1 $ a probabilidade de receber SPAM
    \begin{center}
    \begin{itemize}
      \item $ P(S) = [ C_1 \bigcap S ]\  \bigcup  \ [ C_2 \bigcap S ]\  \bigcup \ [ C_3 \bigcap S ]  $
      \item $ P(S) = P( C_1 \bigcap S ) + P( C_2 \bigcap S ) + P( C_3 \bigcap S )  $
      \item $ P(S) = P(C_1) P( \frac{S}{C_1} ) + P(C_2) P( \frac{S}{C_2} ) + P(C_3) P( \frac{S}{C_3} ) $
      \item $ P(S) = (0,7)*(0,01) + (0,2)*(0,02) + (0,1)*(0,05) $
      \item $ P(S) = 0,0160 \rightarrow 1,6\% $ De se receber um SPAM.
    \end{itemize}
    \end{center}

    \begin{quote}
      2. Sabendo que uma mensagem selecionada aleatoriamente é SPAM qual a probabilidade de que
      ela tenha sido recebida pela conta 3? 
    \end{quote}

    Primeiro redusimos nosso espaço amostral para as mensagens SPAM $ 1,6\% $ e ultilizamos a definição
    de probabilidade 

    \begin{center}
      \begin{large}
        $ P(\frac{C_3}{S}) = \frac{P(C_3 \bigcap S)}{P(S)} \rightarrow  \frac{10\% * 5\%}{1,6\%} = 31,25\% $
      \end{large}
      \end{center}

    Portanto, a probabilidade total é determinada apartir da fórmula
    
    \begin{center}
          $ P(S) = P(E_1) P( \frac{F}{E_1}) + P(E_2) P( \frac{F}{E_2}) + \dots + P(E_k) P( \frac{F}{E_k})  $
    \end{center}
    
    \begin{quote}
      Exemplo 2
    \end{quote}

    Uma doença "rara" acontece 1 em 1000 adultos. Um teste diagnóstifico foi desenvolvido,
    o qual tem o seguinte desempenho:

    \begin{itemize}
      \item Se o indivíduo testado tiver a doença, o teste resulta positivo 99\% das vezes
      \item Se o indivíduo testado \textbf{Não} tiver a doença, o teste resulta positivo 2\% das vezes
    \end{itemize}

    Questão

    \begin{quote}
      1. Se um indivíduo selecionado aleatoriamente foi testado, e o resultado for positivo, qual a 
      probabilidade de ele de fato ter a doença?
    \end{quote}

    Faz ae!

    \begin{center}
    $ P(p) = 4,72\% $ resultado!
    \end{center} 

    \newpage

  \section{Introdução ao teste de Hipóteses (Fundamentos)}

      Exemplo:
  
    \begin{itemize}
      \item[1] Hipótese: A amostra apresentada vem de uma população com média igual a 100
      \item[2] Objeto: Após análise com base em fundamentos estatísticos, decidir
      \subitem a. Rejeitar a hipótese
      \subitem b. Não rejeitar a hipótese
    \end{itemize}

  Calcular \( \longrightarrow \overset{\_}{x} \)
  
  \begin{quote}
    \( \overset{\_}{x} = 103,43 \) 
  \end{quote}

  Se a hipótese nula for verdadeura esperamos encontrar valores de $ \overset{\_}{x} $ próximos de $ \mu $

  %\begin{tikzpicture}
  %  \tzaxes(-2,-3)(2,3){$x$}{$y$}
  %  \tzfn[blue,thick]{(\x)^3}[-1.5:1.5]{$f(x)$}[ar]
  %\end{tikzpicture}

  %\begin{tikzpicture}
  %  \tzaxes(PONTO INICIAL DO EIXO X, PONTO INICIAL DO EIXO Y)( PONTO FINAL DO EIXO X, PONTO FINAL DO EIXO Y ){NOMECLATURA DO EIXO X}{NOMECLATURA DO EIXO Y}
  %  \tzfn[COR DA FUNÇÂO,TIPO DE LINHA]{(FUNÇÃO}[DOMÍNIO DA FUNÇÃO]{NOMECLATURA DA FUNÇÃO}[ar]
  %\end{tikzpicture}
  
  

\end{document}